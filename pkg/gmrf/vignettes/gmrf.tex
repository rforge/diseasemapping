\documentclass[12pt]{article}
\usepackage[margin=1in]{geometry}
\usepackage{amsmath, color}

\DeclareMathOperator{\E}{E}
\DeclareMathOperator{\var}{var}
\DeclareMathOperator{\cov}{cov}
\DeclareMathOperator{\cor}{cor}

\usepackage{filecontents}
\begin{filecontents}{gmrf.bib}
@book{rue2005gaussian,
  title={Gaussian Markov random fields: theory and applications},
  author={Rue, Havard and Held, Leonhard},
  year={2005},
  publisher={CRC Press}
}
@article{offgrid, 
author = {Simpson, D  and Illian, J and  Lindgren, F and S{\o}rbye, S H and
Rue, H},
title={Going off grid:
Computationally efficient inference for log-{G}aussian {C}ox processes.},
note={NTNU Technical report 10/2011 (version 2)},
year={2013},
url={http://arxiv.org/abs/1111.0641v2}
}
@article {mrf,  
author = {Lindgren, Finn and Rue, H{\aa}vard and Lindstr\"{o}m, Johan},
title = {An explicit link between {G}aussian fields and {G}aussian {M}arkov random fields: the stochastic partial differential equation approach},
journal = {Journal of the Royal Statistical Society B },
volume = {73},
number = {4},
publisher = {Blackwell Publishing Ltd},
doi = {10.1111/j.1467-9868.2011.00777.x},
pages = {423--498},
year = {2011}
}

  @BOOK{stein,
  title = {Interpolation of spatial data : some theory for {K}riging},
  publisher = {Springer-Verlag},
  year = {1999},
  author = {Michael Leonard Stein},
  pages = {247},
  address = {New York}
}
  @BOOK{geostats,
  title = {Model-based Geostatistics },
  publisher = {Springer-Verlag},
  year = {2006},
  author = {Peter J Diggle and Paulo Justiniano Ribeiro},
  pages = {230},
  address = {New York},
  isbn = {0387329072}
}
  @misc{wikimatern,
   author = "Wikipedia",
   title = "Mat\'ern covariance function --- {W}ikipedia{,} {T}he {F}ree
   {E}ncyclopedia", 
   year = "2013",
   url =
   "http://en.wikipedia.org/wiki/Matérn_covariance_function",
   note = "[Online; accessed 16-July-2014]"
 }
 @book{cressie,
  title={Statistics for spatial data},
  author={Cressie, Noel},
  year={1993},
  publisher={John Wiley \& Sons}
}
 

\end{filecontents}

\usepackage[authoryear]{natbib}
\bibliographystyle{newapa}

\title{Likelihood for GMRF's}
\author{Patrick Brown}
\begin{document}

\section{Notation}

\citet{rue2005gaussian} wrote an important book, as did \citet{geostats}.

1 is an important number \citep{rue2005gaussian}.

Matern
\[
\rho(h; \phi, \kappa) = \frac{1}{\Gamma(\kappa)2^{\kappa-1}}\left(\frac{\sqrt{8\kappa}||h||}{\phi}\right)^\kappa
K_\kappa\left(\sqrt{8\kappa}||h||/\phi\right).
\]

\begin{itemize}
  \item $\phi$ is range ($\rho$ in \textit{Rue et al})
  \item $\sqrt{8\kappa}/\phi$ is scale ($\kappa$ in \textit{Rue et al})
  \item $\kappa$ is shape ($\nu$ in \textit{Rue et al})
\end{itemize}


First order CAR as
\[
V^{(0)}_i - \frac{\psi}{4}\sum_{j\sim i} V^{(0)}_j =  Z_i \ , \ \
Z_i \sim  \text{i.i.d.\ N}(0, \xi^2\psi/4)
\]
Now define
\begin{align*}
V^{(1)}_i - \frac{\psi}{4}\sum_{j\sim i} V^{(1)}_j = & V^{(0)}_i\\
V^{(2)}_i -\frac{\psi}{4}\sum_{j\sim i} V^{(2)}_j = & V^{(1)}_i
\end{align*}


\begin{itemize}
  \item $\psi$ is the autoregressive parameter, $0\leq\psi < 1$, known as `ar'
  \item $a$ in \textit{Rue et al} is $4/\psi$
  \item $a-4 = 4/\psi-1$
  \item cell size $\delta$
    \item range in cells $ = \phi/ \delta$
  \item scale in cells $=\text{scale}\cdot\delta$
\end{itemize}

\[
	4/\psi = a = (\delta^2\text{scale}^2 + 4) = 8 \kappa \delta^2 / \phi^2 + 4
\]

\[
\psi = \phi^2/(2 \delta^2 \kappa + \phi^2)
\]
\[
\phi =  \delta \sqrt{\frac{2\kappa\psi}{1-\psi}}
\]


$U_i = V^{(\kappa)}_i$


\begin{itemize}
  \item $\text{var}(V^{(0)}_i | V^{(0)}_{-i}) = \xi^2\psi/4$ , $\xi^2$ is conditional
  variance (though strictly speaking conditional variance times 4 divided by
  $\psi$)
  \item $\text{var}(U_i) = \sigma^2$ = marginal variance
\end{itemize}
\[
\sigma^2 =  \xi^2/[4\pi\kappa(a-4)^\kappa] =
\frac{\xi^2}{ 4^{\kappa+1}\pi\kappa} \left(\frac{\psi}{1-\psi}\right)^\kappa
\]


$Q(\psi,\kappa)$ is a precision matrix with $\xi^2=1$, entries as in \textit{Rue et al}

\[
Q(\psi, 1) =
\begin{pmatrix}
1 & &  \\
-8/\psi & 2 & \\
4(1+4/\psi^2) & -8/\psi & 1 \\
\end{pmatrix}
\]

\[
Q(\psi, 2) =
\begin{pmatrix}
-1 & & & \\
12/\psi & -3 & &\\
-3(3+16\psi^2) & 24/\psi & -3& \\
(4/\psi)(12+16/\psi^2) &-3(3+16\psi^2) & 12/\psi & -1& \\
\end{pmatrix}
\]


Precision matrix
\begin{verbatim}
myraster = squareRaster(raster(extent(0,100,0,200), nrow=30, ncol=10))
myPrec = maternGmrfPrec(myraster, param=
	c(conditionalVariance=1, oneminusar=0.05, cellSize = xres(myraster),shape=1)
)

\end{verbatim}



The Method
\begin{itemize}
  \item fix $\psi$ and $\kappa$, derive  $\phi$
  \item For a range of values of $\alpha = \tau^2/\xi^2$, find
  $\hat\xi^2(\psi, \kappa, \alpha)$ and  $\hat\beta(\psi, \kappa, \alpha)$ .
  Note $Q$ does not change with $\alpha$
  \begin{itemize}
    \item ML
    \item REML
    \end{itemize}
	\item Derive $\hat\sigma^2(\psi, \kappa, \alpha)$
  \begin{itemize}
    \item ML
    \item REML
    \end{itemize}
	\item re-do for different $\psi$ and $\kappa$
\end{itemize}


\section*{The Model}
$U$ is a Gaussian Markov random field with sparse precision matrix $Q$.  The MRF
approximation to the Matern specifies the elements of $Q$ as functions of a
range parameter $\phi$.

\begin{align*}
Y_i| U_i \sim & \text{MVN}(X_i\beta + U_i, \tau^2 )\\
[U_1 \ldots U_N]' \sim  & \text{MVN}(0, \xi^2 Q(\phi)^{-1})\\
\var(\mathbf{Y})  = &  ( \xi^2 Q(\phi)^{-1} + \tau^2 I  )  \\
\end{align*}

MLE  $\hat\phi, \hat\xi, \hat\tau,\hat\beta$  maximize $pr(Y ;
\phi,\xi,\tau,\beta)$.

\section*{Likelihood}



\begin{multline*}
-2 \log pr(\mathbf{Y};\phi,\beta,\tau,\xi) =
\log| (\xi^2  Q(\phi)^{-1} + \tau^2 I  ) | + \\
( \mathbf{Y} -\mathbf{X}\beta)  '
  (\xi^2  Q(\phi)^{-1} + \tau^2 I  )^{-1}
(\mathbf{Y} - \mathbf{X}\beta)
\end{multline*}

\[
\hat\beta = [ \mathbf{X}'
(  \xi^2  Q(\phi)^{-1} + \tau^2 I )^{-1}   \mathbf{X}]^{-1}
\mathbf{X}'(\xi^2  Q(\phi)^{-1} +
\tau^2 I  )^{-1} \mathbf{Y}
\]



\begin{align*}
V = & (\xi^2/\tau^2) + Q\\
[\xi^2 Q^{-1} + \tau^2 I]^{-1}= & \tau^2 Q [(\xi^2/\tau^2) + Q]^{-1} \\
= & \tau^2 Q V^{-1}\\
R_X =& Q \mathbf{X}\\
R_Y =& Q \mathbf{Y}\\
V_X = &V^{-1} \mathbf{X}\\
V_Y =& V^{-1} \mathbf{Y}\\
\end{align*}


\begin{align*}
\hat\beta =  &[ \mathbf{X}'
 Q [(\xi^2/\tau^2) + Q]^{-1}  \mathbf{X}]^{-1}
\mathbf{X}' Q [(\xi^2/\tau^2) + Q]^{-1}  \mathbf{Y}\\
= & [ R_X' V_X ]^{-1}  R_X' V_Y\\
\end{align*}


\begin{align*}
-2 \log pr[\mathbf{Y};\phi,\tau, \xi ,\hat\beta(\phi, \xi/\tau)] = &
-\log |\var(\mathbf{Y})^{-1}| + \tau^2 ( \mathbf{Y} -\mathbf{X}\hat\beta) '
  Q V^{-1} 
 [\mathbf{Y} - \mathbf{X}\hat\beta ] \\
R=&  [R_Y' V_Y - R_Y' V_X \hat\beta - \hat\beta ' R_X' V_Y + \hat\beta 'R_X'
V_X \hat\beta ] \\
\hat\beta 'R_X'
V_X \hat\beta = &  \hat\beta' R_X' V_X [ R_X' V_X ]^{-1}  R_X' V_Y =
 \hat\beta ' R_X' V_Y   \\
R = &[R_Y' V_Y - R_Y' V_X \hat\beta] = R_Y' V^{-1} ( \mathbf{Y} -
\mathbf{X}\hat\beta)\\
-2 \log pr[\mathbf{Y};\phi,\tau, \xi ,\hat\beta(\phi, \xi/\tau)] =& -N \log (\tau^2) -\log|Q| + \log|V| +  \tau^2 R\\
\end{align*}

now find $\hat\tau(\tau/\xi, \psi)$
\begin{align*}
\hat\tau^2(\psi, \tau^2/\xi^2) = &  N/R \\
-2 \log pr[\mathbf{Y};\phi,\tau^2/\xi^2, \hat\tau^2(\phi, \tau^2/\xi^2), \hat\beta(\phi, \xi/\tau)] = & -N\log(N) + N \log (R)   -\log|Q| +
\log|V|+N
\end{align*}

\section*{Conditional distributions}

$c = \tau^2/\xi^2$

\begin{align*}
U|Y \sim & \text{MVN}\{\xi^2Q^{-1}(\xi^2 Q^{-1} + \tau^2 I)^{-1})
(\mathbf{Y} - \mathbf{X}\beta), \\
&\qquad
\xi^2Q^{-1}- \xi^2Q^{-1}( \xi^2Q^{-1} + \tau^2 I  )^{-1})\xi^2 Q^{-1} \}\\
\sim &\text{MVN}\left\{  ( I + c Q  )^{-1}  (\mathbf{Y} -
\mathbf{X}\beta),\ \xi^2 Q^{-1}[I- ( I + cQ  )^{-1}	 ] \right\}\\
\end{align*}

getting just the diagonal elements in R
\begin{align*}
\text{diag}\left[Q^{-1} ( I + c Q  )^{-1}\right]	= &
\text{apply}(Q^{-1} \cdot
( I + c Q  )^{-1}, 2,\text{sum})\\
= & \text{apply}(L_2^{-1}\cdot L_2^{-1\prime}Q^{-1},2,\text{sum})
\end{align*}


\section*{Missing values}

\subsection{edge correction}
to do

\subsection{something else}

\appendix

\section{stuff that didnt' work}


\subsection{EM}

\subsubsection{Profile likelihood for $\phi$ (fixed $Q$)}
\begin{multline*}
-2 \log L(U,Y;\xi,\tau,\beta,Q) =
	N\log(\xi^2) + U'Q U/\xi^2 +
	N\log(\tau^2) +\\
	[Y-U-X\hat\beta]'[Y-U-X\hat\beta]/\tau^2
	\end{multline*}
$\hat\beta_{p+1}$ satisfies
\[
\E[ 2  X' X \hat\beta_{p+1} - 2 X'(Y - U) |Y ; p] = 0
 \]
 so
\[
 \hat\beta_{p+1} = (X'X)^{-1}(X'Y) - (X'X)^{-1}(X'\hat U_p)
\]
$\hat\xi^2_{p+1}$ satisfies
\begin{gather*}
N/\xi^2_{p+1} - \text{E}[ U'Q U |Y; p ]/\xi^4 = 0\\
\xi^2_{p+1} = \text{E}[ U'L L' U |Y; p ]/N\\
L'U|Y;p \sim  \text{MVN}\{
	L^{-1}\hat U_p,
\hat\xi_p^2[I - L^{-1}( Q^{-1} + \hat\tau_p^2 I  )^{-1}) L^{-1\prime}] \}\\
\xi^2_{p+1} = (L' \hat U_p )'(L'\hat U_p) + \hat\xi_p^2\left\{N-
\text{trace}\left[(I + \hat\tau_p^2 Q  )^{-1} \right]\right\}
\end{gather*}

$\hat\tau_{p+1}$ satisfies
\begin{gather*}
N /\hat\tau_{p+1} -
\E\{[Y-U-X\hat\beta_{p+1}]'[Y-U-X\hat\beta_{p+1}]/\tau^4|Y;p\}=0\\
\hat\tau_{p+1} = \E\{[Y-U-X\hat\beta_{p+1}]'[Y-U-X\hat\beta_{p+1}]|Y;p\}/N\\
[Y-U-X\hat\beta_{p+1}]'[Y-U-X\hat\beta_{p+1}]=
[Y-X\hat\beta_{p+1}]'[Y-X\hat\beta_{p+1}]
+ U'U -2 [Y-X\hat\beta_{p+1}]' U \\
N \hat\tau_{p+1} = [Y-X\hat\beta_{p+1}]'[Y-X\hat\beta_{p+1}]+
\E[U'U|Y;p] - 2 [Y-X\hat\beta_{p+1}]' \hat U_p\\
\E[U'U|Y;p] = (\mathbf{Y} -
\mathbf{X}\hat\beta_p)' ( I + \hat\tau^2_p/\hat\xi^2_p Q  )^{-2}  (\mathbf{Y} -
\mathbf{X}\hat\beta_p) + \text{trace}\left\{ Q^{-1}\left[\hat\xi_p^2 I-
(\hat\xi_p^2  I + \hat\tau_p^2Q )^{-1} \right]\right\}
\end{gather*}
For a given fixed $\phi$
\begin{itemize}
  \item Start with an initial values  $c_0$
  \item At iteration $p$  do EM with fixed $\phi$ to get $c_{p+1}$
  \item repeat to get $\hat c(\phi)$
  \item Calculate likelihood
  $L(Y;\phi,\hat c(\phi), \hat\beta[\phi,\hat c(\phi)) $
\end{itemize}
Note only one cholesky required per iteration when $\phi$ is fixed.

\subsubsection*{MLE's only}
For a given fixed $\phi$
\begin{itemize}
  \item Start with initial values  $\phi_0$, $c_0$
  \item At iteration $p$  to get $\phi_{p+1}$, $c_{p+1}$
  \item repeat to get $\hat c(\phi)$
\end{itemize}
Note two cholesky's per iteration. Need cholesky of $Q$ to get $\hat\beta$



\section*{The old likelihood}
\begin{multline*}
-2 \log pr(\mathbf{Y};\phi,\beta,\tau) =
| (\xi^2  Q(\phi)^{-1} + \tau^2 I  ) | + \\
( \mathbf{Y} -\mathbf{X}\beta)  '
  (\xi^2  Q(\phi)^{-1} + \tau^2 I  )^{-1}
(\mathbf{Y} - \mathbf{X}\beta)
\end{multline*}

\[
\hat\beta = [ \mathbf{X}'
(  \xi^2  Q(\phi)^{-1} + \tau^2 I )^{-1}   \mathbf{X}]^{-1}
\mathbf{X}'(\xi^2  Q(\phi)^{-1} +
\tau^2 I  )^{-1} \mathbf{Y}
\]

Some tricks
\begin{align*}
 Q(\phi) = L L',\   Q(\phi)^{-1} &=L^{\prime-1} L^{-1}, \ \  c = \xi^2/\tau^2 \\
\tilde Y = L' Y,&  \  \tilde X = L' X\\
\end{align*}
\begin{align*}
\var(\mathbf{Y}) =&  [ \xi^2 Q(\psi)^{-1} +  \tau^2 I] \\
= &    L^{\prime -1}(  \xi^2I +   \tau^2 L'L  ) L^{-1}\\
c = & \tau^2/\xi^2\\
\var(\mathbf{Y})^{-1} = &    L(   I/c +   L'L   )^{-1} L' / \tau^2\\
L_2 L_2' = & I/c +    L'L \\
\var(\mathbf{Y})^{-1} = &    L L_2^{-1\prime} L_2^{-1} L' / \tau^2\\
\end{align*}


Note that $|aX| = a^N|X|$ where $X$ is $N$ by $N$
\begin{align*}
\log |\var(\mathbf{Y})| = & N \log(\tau^2) + 2 \log |L_2| - 2 \log |L|\\
\end{align*}



 
\begin{align*}
\hat\beta(\phi, \xi^2/\tau^2) = &[ \mathbf{X}' 
\var(\mathbf{Y})^{-1}  \mathbf{X}]^{-1}  
\mathbf{X}'\var(\mathbf{Y})^{-1} \mathbf{Y}\\
 = &[ \mathbf{X}' 
  L  L_2^{-1\prime} L_2^{-1}  L'    \mathbf{X}]^{-1}  
\mathbf{X}' L  L_2^{-1\prime} L_2^{-1}   L' \mathbf{Y}
\\
\breve Y =& L_2^{-2} L'  \mathbf{Y}\\
\breve X =& L_2^{-2} L' \mathbf{X}\\
\hat\beta(\phi, \xi^2/\tau^2) = & (\breve X' \breve X)^{-1} \breve X' \breve Y
\\
-2 \log pr(\mathbf{Y};\phi,\tau, \xi) = &
\log |\var(\mathbf{Y})| + ( \mathbf{Y} -\mathbf{X}\hat\beta[\phi, \xi^2/\tau^2) ]  '   
  \var(\mathbf{Y}) ^{-1} 
[\mathbf{Y} - \mathbf{X}\hat\beta(\phi, \xi^2/\tau^2) ] \\
R = & L_2^{-1}   L' [\mathbf{Y} - \mathbf{X}\hat\beta(\phi, \xi^2/\tau^2) ] \\
= & \breve Y - \breve X \hat\beta(\phi, \xi^2/\tau^2) \\
-2 \log pr(\mathbf{Y};\phi,\tau, \xi)  = &  - \log|  \tau^2  L(
L_2^{-1\prime} L_2^{-1} ) L'| +  \tau^2 R'R\\
= & - N \log(\tau^2) + 2\log|L_2| - 2\log|L| +  \tau^2 R'R\\
\hat\tau(\psi, \tau^2/\xi^2) = &  N/R'R \\
-2 \log pr(\mathbf{Y};\phi,\tau^2/\xi^2)+C = & - N \log (R'R)   + 2\log|L_2| -
2\log|L| 
\end{align*}




 


The algorithm
\begin{itemize}
  \item For an initial range parameter $\phi$, construct $Q(\phi)$, $LL'$,
  $\tilde Y$ and $\tilde X$
  \item For for each $\phi$, compute $L_2$, $\breve X$ and $\breve Y$ for a
  range of $c$, find $\hat c(\phi)$
  \item Repeat for multiple $\phi$ to get $\hat \phi$
\end{itemize}


\begin{align*}
Y = & X \beta + Z U + Z\\
Z \sim & \text{N}(0, \tau^2I )\\
[U_1 \ldots U_M]' \sim  & \text{MVN}(0, \xi^2 Q(\phi)^{-1})\\
\var(\mathbf{Y})  = &  ( \xi^2 Z' Q(\phi)^{-1}Z)  + \tau^2 I    \\
\end{align*}

\begin{itemize}
\item First Cholesky  $Q = L_1 L_1'$ and backsolve $W = L_1^{-1}Z$
\item Second Cholesky $\var(Y) = W'W + c I = L_2 L_2'$
\item Backsolve $L_2^{-1}X$ and  $L_2^{-1}Y$ to get $\hat\beta$
\end{itemize}

\begin{align*}
E(U|Y) = &  Q(\phi)^{-1}
(\xi^2 Z' Q(\phi)^{-1}Z + c I)^{-1})
(\mathbf{Y} - \mathbf{X}\hat\beta)\\
E(Z|Y) = & \tau^2 I(\xi^2 Z' Q(\phi)^{-1}Z + c I)^{-1})
(\mathbf{Y} - \mathbf{X}\hat\beta)
\end{align*}

\begin{itemize}
  \item
\end{itemize}
\begin{align*}
\var(\mathbf{Y}) = &  \xi^2  Z ' L^{\prime -1}(  I + c  Q(\phi)  ) L^{-1} Z\\
\var(\mathbf{Y})^{-1} = &    L(  I + c  Q(\phi)  )^{-1} L' / \xi^2\\
\log |\var(\mathbf{Y})| = & N \log(\xi^2) + \log |I + c Q(\phi)| -
2 \log |L|\\
\end{align*}

\bibliography{gmrf}

\end{document}

