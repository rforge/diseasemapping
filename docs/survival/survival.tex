\documentclass{article}
\usepackage{amsmath}
\usepackage[margin=1in]{geometry}

\begin{document}
\section{The model}
Person $i, i=1\ldots N$ lives at location $s_i$, and is not susceptible to cancer (cured) if $Y_i=1$.  If they are susceptible, they have a cancer incidence time $T_i$.
\begin{align*}
Y_i & \sim \text{Bernoulli}(\rho)\\
[T_i | Y_i = 0] &\sim \text{Weibull}(\gamma_i, \nu)\\
[T_i | Y_i = 1] & = \infty \\
\log(\gamma_i) &= X_i \beta + U(s_i)\\
\text{cov}[U(s_0 + s), U(s_0)] &= \sigma^2 M(s; \theta, \kappa) 
\end{align*}
$\gamma_i$ and $\nu$ are the range and shape parameters of the Weibull respectively, and $\rho$ is the cure fraction.  $U(s)$ is a Gaussian random field with a Matern correlation function with scale parameter $\theta$ and roughness $\kappa$.  In practice $U(s)$ will be approximated by a Markov random field $\bar U$ on a grid $g_1 \ldots g_M$ with $U(g_k) \approx \bar U_k$.



The density of a Weibull with range parameter $\gamma$ and shape parameter $\nu$ is 
\[
f(x;\gamma,\nu) =  \gamma \nu x^{\nu-1} \exp\left[-	\gamma x^\nu\right],
\]
and
\[
\int_x^\infty f(u;\gamma,\nu)du = \exp(-\gamma x^\nu).
\]
If $\theta$ is the scale parameter in the parametrization of the Weibull in Wikipedia, then 
$\log(\gamma) = -\alpha\log(\theta)$


\subsection{Notation}

Each individual data for one or more of the following variables:
\begin{itemize}
\item Time of event $T_i$, often unobserved
	\item left trunCation time $C_i$, with $C_i=0$ for untruncated observations
\item Lower bound for the event time $L_i$
\item Upper bound $U_i$
\item Event type $E_i$ defined as
\begin{itemize}
	\item $E_i=1$: event observed at time $T_i$
	\item $E_i=0$: event right censored at $L_i$, with $T_i$ unobserved but $T_i >L_i$.
	\item $E_i=2$: left censored data at $U_i$, with $T_i < U_i$.
	\item $E_i=3$: interval censored data with $L_i < T_i < U_i$. 
\end{itemize}
\end{itemize}
The 'cure' indicator $Y_i$ is not part of the dataset.  Note that $E_i$ isnt strictly necessary since it can be worked out by which of $T_i$, $L_i$ and $U_i$ are specified.

\section{Likelihood}

\subsection{Right censored observations $E_i =0$}

\begin{align*}
pr(L_i, E_i=0 | C_i) =& pr(T_i > L_i | T_i > C_i, Y_i =0) pr(Y_i =0)  + pr(Y_i =1)\\
=&pr(T_i > L_i) pr(Y_i = 0)/pr(T_i > C_i) + pr(Y_i =1)\\
=& \rho + (1-\rho) \left. \int_{L_i}^\infty f(x;\gamma_i, \nu) dx \right/\int_{C_i}^\infty f(x;\gamma_i, \nu) dx  \\
= &  \rho + (1-\rho) \left. \exp\left[-\gamma_i L_i^\nu\right] \right/\exp\left[-\gamma_i C_i^\nu\right]\\
= &  \rho + (1-\rho) \exp\left[\gamma_i\left(C_i^\nu -L_i^\nu\right)\right] \\
\end{align*}
Note that $C_i=0$ implies the $C_i^\nu$ term can be omitted and the untruncated likelihood results.

\subsection{Observed events $E_i =1$}
\begin{align*}
pr(T_i, E_i=1 | C_i) =& pr(T_i  | T_i > C_i, Y_i =0)  pr(Y_i = 0) 
\\
=& 
 pr(T_i ) pr(Y_i = 0) /pr(T_i > C_i) \\
=&
 (1-\rho) f(T_i ; \mu_i, \nu) \left/ \int_{C_i}^\infty f(x;\mu_i, \nu) dx  \right.\\
=&
 (1-\rho) \gamma_i\nu T_i^{\nu-1} \exp\left[\gamma\left(C_i^\nu-T_i^\nu\right)\right]
\end{align*}

\subsection{Left censored observations $E_i =2$}
I'm a bit confused about this one...  For completeness we should define the likelihood for left censoring with left truncation.
\begin{align*}
pr(U_i, E_i=2 | C_i) =& pr(C_i < T_i < U_i | T_i > C_i, Y_i =0) pr(Y_i =0) \\
=&pr(C_i < T_i <U_i) pr(Y_i = 0)/pr(T_i > C_i) \\
=& (1-\rho) \left. \int_{C_i}^{U_i} f(x;\gamma_i, \nu) dx \right/\int_{C_i}^\infty f(x;\gamma_i, \nu) dx  \\
= &    (1-\rho) \left. \left[\exp(-\gamma_i C_i^\nu) - \exp(-\gamma_i U_i^\nu) \right] \right/\exp\left[-\gamma_i C_i^\nu\right]\\
= &  (1-\rho)\left\{1 -  \exp\left[\gamma_i\left(C_i^\nu -U_i^\nu\right)\right] \right\} \\
\end{align*}



\subsection{Interval censored events, $E_i =2$}


\begin{align*}
pr(L_i < T_i < U_i | T_i > C_i ) =& (1-\rho) \int_{L_i}^{U_i}f(u;\gamma_i, \nu) du 
\\
=& (1-\rho) \left[ \exp(-\gamma L_i^\nu) -  \exp(-\gamma U_i^\nu)\right] / \exp\left[-\gamma_i C_i^\nu\right]
\\
\end{align*}
The denominator is 1 for untrancated $C_i =0$.



\end{document}